%期末展示beamer正在修改...
\documentclass{beamer}

\usepackage{threeparttable}
\usepackage{booktabs}
\usepackage{graphicx}
\usepackage[UTF8]{ctex}


\author{Hongming Liang\\  2022201480}
\title{Linear Model For House Pricing}
\date{2025/04/03}

\begin{document}

\frame{\titlepage}

\begin{frame}{Scaling}
    \begin{figure}
        \begin{minipage}[t]{0.3\textwidth}
            \center
            \includegraphics[width=\linewidth]{Figures/PriceBoxPlot.jpg}
            \caption{\scriptsize BoxPlot for $price$}
        \end{minipage}
        \begin{minipage}[t]{0.3\textwidth}
            \center
            \includegraphics[width=\linewidth]{Figures/PriceDistribution.jpg}
            \caption{\scriptsize Distribution for $price$}
        \end{minipage}
        \begin{minipage}[t]{0.3\textwidth}
            \center
            \includegraphics[width=\linewidth]{Figures/LnPriceDistribution.jpg}
            \caption{\scriptsize Distribution for $\ln price$}
        \end{minipage}
    \end{figure}
    \begin{figure}
        \begin{minipage}[t]{0.3\textwidth}
            \center
            \includegraphics[width=\linewidth]{Figures/AreaBoxPlot.jpg}
            \caption{\scriptsize BoxPlot for $area$}
        \end{minipage}
        \begin{minipage}[t]{0.3\textwidth}
            \center
            \includegraphics[width=\linewidth]{Figures/LnAreaDistribution.jpg}
            \caption{\scriptsize Distribution for $\ln area$}
        \end{minipage}
        \begin{minipage}[p]{0.25\textwidth}
            \scriptsize
            \begin{table}
                \center
                \begin{tabular}{c|cc}
                    \toprule
                    Varible & $price, area$ & $\ln()$ \\
                    \midrule
                    MAE & 176341 & 156847 \\
                    $R^2$ & 0.8559 & 0.9671 \\
                    Score & 58.12 & 81.00 \\
                    \bottomrule
                \end{tabular}
                \caption{\scriptsize Comparison Using OLS (Out-of-sample)}
            \end{table}
        \end{minipage}
    \end{figure}
\end{frame}

\begin{frame}{Features}
    \begin{itemize}
        \item Numeric features: '建筑面积', '梯户比例', '绿 化 率', '停车位'。其中'梯户比例'提取字符串中的数值计算得到
        \item Categorical features: '板块', '城市', '环线', '房屋户型', '所在楼层', '小区名称', '建筑结构', '装修情况','房屋朝向', '配备电梯', '别墅类型', '交易权属', '房屋用途', '房屋年限', '产权所属', '区域'
        \item Special Care: \begin{itemize}
            \item '环线':创建城市与环线的交乘项。不同城市的环线存在异质性
            \item '房屋朝向':提取对应的字符串,创建了八个虚拟变量,即是否朝向'东','南','西','北','东北','东南','西南','西北'八个方向
            \item '房屋户型':提取字符串中的房间个数用于创建虚拟变量'n\_室','n\_厅','n\_厨','n\_卫'
            \item 为了充分利用信息,类别特征的缺失值统一用'未知'来填补,例如'配备电梯'这一特征。
        \end{itemize}
    \end{itemize}
\end{frame}

\begin{frame}{Scores}
    \small
    \begin{itemize}
        \item Model:$$
        \ln price_i = \beta_0 + \beta_1 \ln area_i + \gamma  X_i + \delta _i + \epsilon_i
        $$
    \end{itemize}
    \begin{threeparttable}
        \center
        \caption{MAE for OLS, Lasso and Ridge}
        \begin{tabular}{c|cccc}
            \toprule
            Metrics & In sample & Out of sample & Cross-validation & Datahub Score \\
            \midrule
            OLS & 143464 & 156847 & 1766221 & 81.00 \\
            Lasso & 238477 & 236534 & 1733797 & 79.55\\
            Ridge & 143683 & 156802 & 1766258 & 80.99\\
            Best & 143464 & 156847 & 1766221 & 81.00\\
            \bottomrule
        \end{tabular}
        \begin{tablenotes}
            \item Note: The best model is OLS. Lasso with parameter $\alpha = 10^{-5}$. Ridge with parameter $\alpha = 0.01$. 1284 outliers were removed.
         \end{tablenotes}
    \end{threeparttable}
\end{frame}

\end{document}